\documentclass[12pt]{article}
\usepackage[utf8]{inputenc}
\usepackage{amsmath, amssymb}
\usepackage{geometry}
\usepackage{fancyhdr}
\usepackage{tikz}
\usepackage{array}
% define the geometry of the document.
\geometry{
    a4paper,
    total={170mm,257mm},
    left=20mm,
    top=20mm,
    right=20mm,
    bottom=20mm
}

\makeatletter
\def\borderline{
    \begin{tikzpicture}[remember picture,overlay]
    \draw[line width=1pt] 
    ($(current page.north west)+(0.75cm,-0.75cm)$) 
    rectangle 
    ($(current page.south east)-(0.75cm,0.75cm)$);
    
    \draw[line width=0.4pt,densely dotted] 
    ($(current page.north west)+(1.25cm,-1.25cm)$) 
    rectangle 
    ($(current page.south east)-(1.25cm,1.25cm)$);
    \end{tikzpicture}
}
\makeatother

\pagestyle{fancy}
\fancyhf{}
\renewcommand{\headrulewidth}{0pt}
\fancyfoot[C]{\thepage}

\begin{document}

\begin{center}
    \renewcommand{\arraystretch}{1.5}
    \begin{tabular}{ | m{5cm} | m{7cm} | }
        \hline
        \textbf{Name} & Sphamandla Mbuyazi \\
        \hline
        \textbf{Student Number} & 2618115 \\
        \hline
    \end{tabular}
\end{center}

\borderline

\section*{Question 1}
To find the value of $c$ such that $f(x)$ is a distribution, we ensure that the total probability integrates to 1:

\begin{align*}
    \int_0^6 c(6-x) dx &= 1 \\
    c \int_0^6 (6-x) dx &= 1 \\
    c \left[6x - \frac{x^2}{2}\right]_0^6 &= 1 \\
    18c &= 1 \\
    c &= \frac{1}{18}
\end{align*}

\subsection*{Mean}
\begin{align*}
    E[X] &= \int_0^6 x \cdot \frac{1}{18}(6-x) dx \\
         &= \frac{1}{18} \int_0^6 (6x - x^2) dx \\
         &= \frac{1}{18} \left[3x^2 - \frac{x^3}{3}\right]_0^6 \\
         &= 2
\end{align*}

\subsection*{Variance}
\begin{align*}
    E[X^2] &= \frac{1}{18} \int_0^6 x^2(6-x) dx \\
           &= \frac{1}{18} \left[2x^3 - \frac{x^4}{4}\right]_0^6 \\
           &= 6
\end{align*}
\begin{align*}
    V(X) &= E[X^2] - (E[X])^2 \\
         &= 6 - 2^2 = 2
\end{align*}

\section*{Question 2}
Let $X \sim U(-2, 3)$, the uniform distribution between -2 and 3.

\subsection*{(a) $P(X > 1)$}
\begin{align*}
    P(X > 1) &= \frac{3 - 1}{5} = 0.4
\end{align*}
% this is because solving this equation gives [-1,1]; and we want when x is less than -1 and greater than 1
\subsection*{(b) $P(X^2 > 1)$}
\begin{align*}
    P(X^2 > 1) &= P(X < -1 \text{ or } X > 1) \\
    &= \frac{1 + 2}{5} = 0.6
\end{align*}

\subsection*{(c) $E[X]$}
\begin{align*}
    E[X] &= \frac{-2 + 3}{2} = 0.5
\end{align*}

\subsection*{(d) $V(X)$}
\begin{align*}
    V(X) &= \frac{(3 - (-2))^2}{12} = \frac{25}{12}
\end{align*}

\section*{Question 3}
Let $X \sim \text{Exponential}(\lambda = 1.5)$.

\subsection*{(a) $P(X > 2)$}
\begin{align*}
    P(X > 2) &= e^{-1.5 \cdot 2} = e^{-3} \approx 0.0498
\end{align*}

\subsection*{(b) $P(X > 4)$}
\begin{align*}
    P(X > 4) &= e^{-1.5 \cdot 4} = e^{-6} \approx 0.00248
\end{align*}

\subsection*{(c) $E[X]$}
\begin{align*}
    E[X] &= \frac{1}{1.5} \approx 0.667
\end{align*}

\subsection*{(d) $V(X)$}
\begin{align*}
    V(X) &= \frac{1}{(1.5)^2} \approx 0.444
\end{align*}

\end{document}
